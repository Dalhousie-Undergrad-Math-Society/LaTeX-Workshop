\documentclass{article}

\usepackage{amsmath, amsthm, amssymb, amsfonts, lipsum, hyperref}
\setlength{\parindent}{0pt}
\setlength{\parskip}{\baselineskip}

\topmargin=-0.5in
\evensidemargin=0in
\oddsidemargin=0in
\textwidth=6.5in
\textheight=9.0in
\headsep=0.25in
\linespread{1.1}

\title{
    \textbf{Polynomials, Powerful Tangents, and Permutations}
    \author{Peter Piper}
}
\makeatletter
\renewcommand*\env@matrix[1][*\c@MaxMatrixCols c]{%
  \hskip -\arraycolsep
  \let\@ifnextchar\new@ifnextchar
  \array{#1}}
\makeatother


\begin{document}
\maketitle
\pagebreak

\section*{Problem 1}
\textit{Given real numbers, $a,b,c \in \mathbb{R},$ the polynomial \[g(x) = x^3 + ax^2 + x + 10,\] is such that all of its roots are also roots of the polynomial \[
f(x) = x^4 + x^3 + bx^2 + 100x + c.
\] What is $f(1)$? (2017 AMC 12A Problem 23)}


\textbf{Claim:} $f(1) = -7007$
\begin{proof}
    Let $r_1, r_2, r_3$ be the roots of $g(x)$, and  $r_4$ be the additional root of $f(x)$. By Vieta's Formulas, we have that  
    \begin{align*}
        r_1 + r_2+r_3 &= -a\\
        r_1+r_2+r_3+r_4 &= -1
    \end{align*}
    so we know that $r_4 = a - 1$.

    Vieta's formulas also tell us that 
    \begin{align*}
        r_1 r_2 r_3 &= -10\\
        r_1 r_2 + r_2 r_3 + r_3 r_1 &= 1\\ 
        r_1r_2r_3 + r_2r_3r_4 + r_2r_4r_1 &= -100 \label{eq:1}\tag{$\star$}
    \end{align*}
Substituting $r_1r_2r_3 = -10$ into \ref{eq:1} gives us that \[
    -10 + (r_1r_2 + r_2r_3 + r_3r_1)r_4 = -10 + r_4 = -100.
    \] Then $r_4 = 90$, so $a = 89$.

    By factoring  $f(x)$ in terms of  $g(x)$, we get  \[
    f(x) = (x-r_4)g(x) = \left(x + 90\right) g(x)
    \] 
    Then since $g(1) = -77$,  $f(1) = 91 \cdot -77 = -7007$, as desired.
\end{proof}

\pagebreak

\section*{Problem 2}
\textit{There is a unique $\theta$ between  $0^\circ$ and $90^\circ$ such that for nonnegative  $n \in \mathbb{Z}$, $\tan(2^n \theta)$ is positive if and only if  $n \equiv_3 0$. If  $\theta = \frac{p}{q}$ for two relatively prime integers, find  $p + q$. (2019 AIME II Problem 10)}

\begin{proof}
    Note that if $\tan(\theta)$ is positive, then $0^\circ < \theta < 90^\circ \mod{180}$. Furthermore, it must also hold that \[
        2^0 \theta \equiv 2^3 \theta \equiv 2^6 \theta \equiv \dots \mod{180}
    \] as if it did not, the terminal angle would shift out of the first quadrant.

    Then, $2^3 \theta \equiv 2^0 \theta$ so  $7 \theta \equiv 0^\circ \mod{180}$ Thus, $\theta$ must be one of  \[
        \frac{180^\circ}{7}, \quad \frac{360^\circ}{7}, \quad \frac{540^\circ}{7}.
    \] The only value of these three that works is $\theta = \frac{540^\circ}{7}$, and these two integers are already coprime, so our answer is $547$.
\end{proof}

\pagebreak

\section*{Problem 3}
\textit{Prove the Hockey-Stick Identity, \[
        \sum^n_{i = r} \binom{i}{r} = \binom{n+1}{r+1}
\] for all $n, r \in  \mathbb{N}, n > r$}

\begin{proof}
    Let $n = r$. Then we have that 
     \[
         \sum_{i=r}^{n} \binom{i}{r} = \sum^r_{i=r} \binom{i}{r} = \binom{r}{r} = 1 = \binom{r+1}{r+1}.
    \] 
    Then let $k \in \mathbb{N}$ such that $k > r$ and  \[
        \sum_{i = r}^{k} \binom{i}{r} = \binom{k+1}{r+1}.
    \] Then we have
    \begin{align*}
        \sum^{k+1}_{i=r} \binom{i}{r} &= \left( \sum^k_{i=r} \binom{i}{r} \right) + \binom{k+1}{r}\\
    &= \binom{k+1}{r+1} + \binom{k+1}{r} \\ 
    &= \binom{k+2}{r+1}
    \end{align*} as desired.
    
\end{proof}
\pagebreak

\section*{Problem 4}
\textit{Find the eigenvalues and eigenvectors of \[
        A = \begin{bmatrix} 0 & -1 \\ 1 & 0 \end{bmatrix} 
\] over $\mathbb{C}$. Diagonalize if possible.}

\begin{proof}
    The characteristic polynomial of $A$ is $\lambda^2 + 1$. This has roots in $\mathbb{C}$, being  $\lambda = \pm i$. We now solve for the eigenvectors.

    \textbf{Case 1: $\lambda = i$} 

    We solve $(A - iI)v = 0.$ 

        \[
            \begin{bmatrix}[cc | c]
                -i & -1 &  0\\ 
                1 & -i & 0 \\
            \end{bmatrix} 
            \simeq 
            \begin{bmatrix}[c c | c]
                1 & -i & 0 \\ 
                -i & -1 & 0
            \end{bmatrix} 
            \simeq 
            \begin{bmatrix}[c c | c]
                1 & -i & 0 \\ 
                0 & 0 & 0 \\
            \end{bmatrix} 
        \] Thus, our eigenvector for $\lambda = -i$ is  \[
        v_1 = \begin{bmatrix} 
            i \\ 
            1
        \end{bmatrix}. 
        \] 

        We solve for $\lambda = i$ similarly to find  \[
        v_2 = \begin{bmatrix} -i \\ i  \end{bmatrix}.
        \] 

        Then since we have two linearly independent eigenvectors, $A$ is diagonalizable and  $A = P^{-1} D P $ where  \[
            P = \begin{bmatrix} i & -i \\ 1 & 1 \end{bmatrix} \quad D = \begin{bmatrix} i & 0 \\ 0 & -i \end{bmatrix}.
        \] 
    
\end{proof}

\end{document}
